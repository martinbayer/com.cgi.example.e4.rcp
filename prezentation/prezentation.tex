\documentclass{beamer}
\usepackage[utf8]{inputenc}
\usepackage{bbding}

\usetheme{cgi}

\title{Modularity with Eclipse RCP}
\author{Martin Bayer \& Tom\'{a}\v{s} Vejpustek}

\makeatletter%<<<
\newcommand{\@eg}{\raisebox{-0.5mm}{\HandRight}}
\def\eg{\@ifstar{\textcolor{ice}{\@eg}}{\textcolor{pumpkin}{\@eg}}}
\newcommand{\fig}{\textcolor{pumpkin}{\raisebox{-0.5mm}{\PencilRightDown}}}
\newcommand{\str}[1]{\structure{#1}}

\newcommand{\breakframe}{\begin{fillerframe}\frametitle{Break: Questions?}\end{fillerframe}}
\newenvironment{centerblock}{\begin{block}{}\begin{center}}{\end{center}\end{block}}
\newcommand{\ttt}[1]{\texttt{\color{honey}#1}}
\newcommand{\mtt}[1]{\ensuremath{\mathtt{#1}}}
\newcommand{\singleindent}[1]{\begin{itemize}\item[] #1\end{itemize}}
\let\oldurl\url
\renewcommand{\url}[1]{\textcolor{pumpkin}{\scriptsize\oldurl{#1}}}
\makeatother%>>>

\begin{document}

\titlepageframe
\begin{frame}{Outline}%<<<
	Basic understanding of RCP:
	\begin{enumerate}
		\item What is Eclipse RCP?
		\item Dependency injection, inversion of control
		\item Application architecture
		\item Communication
		\item Modularity with plugins
	\end{enumerate}
\end{frame}%>>>
\begin{frame}{Eclipse Rich Client Platform}%<<<
	\begin{itemize}
		\item \alert{not} plugins for Eclipse IDE
		\item software framework
		\item thick clients with GUI
		\item layout looks like Eclipse IDE \eg
	\end{itemize}
	%TODO example picture or notice
\end{frame}%>>>
\begin{frame}{Why RCP}
	% TODO nad tímhle chce ještě zapracovat
\end{frame}
\begin{frame}{Dependency Injection}%<<<
	\begin{itemize}
		\item[=] means of accessing software libraries \fig
			\begin{enumerate}
				\item I~need \str{service}
				\item I~ask \str{framework} for \str{interface}
				\item \str{framework} finds (and initializes) \str{implementation}
			\end{enumerate}

		\item\str{implementations} are interchangeable
	\end{itemize}
\end{frame}%>>>
\begin{frame}{Dependency Injection in RCP}%<<<
	\begin{itemize}
		\item based on \str{OSGi}
		\item annotations \eg
			\begin{itemize}
				\item fields, constructors -- \ttt{@Inject}
				\item methods -- behavioral (\ttt{@PostConstruct}, \ldots)
			\end{itemize}
	\end{itemize}

	\begin{block}{Context}
		\begin{itemize}
			\item[=] storage for injectable objects
			\item[=] mechanism of searching injectable objects
		\end{itemize}
		\begin{center}``Let's put this object into context''\end{center}
	\end{block}
\end{frame}%>>>
\begin{frame}{Pros and Cons of Dependency Injection}%<<<
	\begin{itemize}
		\item[+] \str{independence on concrete implementation}
		\item[+] increased modularity
		\item[+] easy testing
		\item[+] independent development (only need API)
		\item[+] reduces boilerplate dependency obtaining
		\item[$-$] difficult to trace and debug (\alert{document well!})
	\end{itemize}
\end{frame}%>>>
\begin{frame}{Inversion of Control}%<<<
	\begin{centerblock}``Don't call us, we'll call you.''\end{centerblock}
	\begin{itemize}
		\item[$\approx$] software framework
		\item DI is one implementation
		\item user writes \str{snippets} and \str{framework calls} them
		\item user writes \str{GUI elements} and \str{framework places} them
		\item when/where $\Leftarrow$ configuration (XML files)
	\end{itemize}
\end{frame}%>>>
\begin{frame}{Application Model}%<<<
	\begin{itemize}
		\item basis of RCP framework
		\item Java classes assigned to elements
		\item behaviour ruled by annotations
		\item elements \eg
			\begin{itemize}
				\item graphical: window, perspective, part
				\item commands, handlers, menus, tool items
				\item addons (provide services)
			\end{itemize}
		\item can be changed dynamically (\ttt{EModelService}) \eg*
	\end{itemize}
\end{frame}%>>>
\begin{frame}{Commands and Handlers}%<<<
	\begin{description}
		\item[Command] abstract action (save, \ldots)
		\item[Handler] actual implementation
	\end{description}

	\bigskip

	\begin{itemize}
		\item DI in practice \fig \eg
			\begin{enumerate}
				\item button calls \str{command}
				\item \str{framework} finds closest \str{handler}
			\end{enumerate}
		\item different ``save'' for different editors
		\item handlers can be called from code \eg*
	\end{itemize}
\end{frame}%>>>

\breakframe

\begin{frame}{Communication}%<<<
	\begin{centerblock}How to pass information/objects between modules/plugins?\end{centerblock}

	\medskip

	Communication in RCP:
	\begin{itemize}
		\item DI-based
		\item[$-$] weird, hard-to-trace errors (missing producer)
		\item[$+$] great decoupling
	\end{itemize}
\end{frame}%>>>
\begin{frame}{Eclipse Context}%<<<
	\begin{itemize}
		\item basis of DI in RCP
		\item storage of objects
	\end{itemize}
	\begin{centerblock}\begin{tabular}{rl}
		used as & $\mtt{Class\langle T\rangle}\to\mtt{T}$\\
		implemented as & $\mtt{String} \to \mtt{Object}$ (\ttt{@Named})\\
	\end{tabular}\end{centerblock}
	Passing objects \eg
	\begin{itemize}
		\item send using \ttt{IEclipseContext} (from DI)
		\item receive via DI
	\end{itemize}
\end{frame}%>>>
\begin{frame}{Context Hierarchy}%<<<
	\begin{itemize}
		\item own context for \str{app model} elements -- \ttt{MContext}\\
			\singleindent{application, window, perspective, part, popup menu}

		\item DI look-up \fig
			\begin{enumerate}
				\item search current context
				\item not found $\Rightarrow$ search parent context; \alert{repeat}
				\item not found in application context $\Rightarrow$ \texttt{null}/error
			\end{enumerate}

		\item used in handler look-up
		\item \alert{Use common context for communication!} \eg
	\end{itemize}
\end{frame}%>>>
\begin{frame}{Event Service}%<<<
	\begin{itemize}
		\item global messages (context-independent)
		\item stringly typed (message topics)
	\end{itemize}

	\medskip

	Passing objects \eg
	\begin{itemize}
		\item send using \ttt{IEventBroker} (from DI)
			\singleindent{\ttt{send} (sync), \ttt{post} (async)}
		\item receive via DI
			\singleindent{\ttt{@EventTopic}, \ttt{@UIEventTopic} (UI thread)}
	\end{itemize}
\end{frame}%>>>
\begin{frame}{Context vs Event Service}%<<<
	\begin{itemize}
		\item \str{persistence} vs \str{notification}
		\item persistence important for inactive parts
	\end{itemize}

	\begin{centerblock}\begin{tabular}{l@{\ $\Rightarrow$\ }l}
		I~need to be notified & use \str{Event Service}\\
		I~need to retrieve it later & use \str{Context}\\
		I~need both & use both \eg\\
	\end{tabular}\end{centerblock}

	\begin{itemize}
		\item use \str{context} for parts and handlers
		\item handlers do not need notification \alert{[!]}
	\end{itemize}
\end{frame}%>>>
\begin{frame}{Selection Service}%<<<
	\begin{itemize}
		\item context + event service
		\item active GUI selection
		\item window and part -specific
	\end{itemize}
	Passing objects \eg
	\begin{itemize}
		\item send using \ttt{ESelectionService.setSelection}
		\item receive via DI: \ttt{@Named(IServiceConstants.ACTIVE\_SELECTION)}
	\end{itemize}
\end{frame}%>>>

\breakframe

\begin{frame}{Plugin}%<<<
	\begin{itemize}
		\item[=] application module
			\begin{enumerate}
				\item describes what it provides/extends
				\item framework plugs it in
			\end{enumerate}
		\item defined in \ttt{plugin.xml}
		\item dependencies and provided packages in \ttt{MANIFEST.MF}
		\item plugin $\subseteq$ feature $\subseteq$ product
	\end{itemize}
\end{frame}%>>>
\begin{frame}{Extensions}%<<<
	\begin{itemize}
		\item built on \str{OSGi} extensions (simplified)
	\end{itemize}
	\begin{enumerate}
		\item define \str{extension point} -- contract for \str{extensions}
			\begin{itemize}
				\item attributes -- primitive types (XML Schema)
				\item executable extension -- implements/extends
			\end{itemize}
		\item register contributing extensions (\ttt{plugin.xml})
		\item process contributions (\ttt{IExtensionRegistry} via DI)
	\end{enumerate}\eg
\end{frame}%>>>
\begin{frame}{Executable Extensions}%<<<
	\begin{itemize}
		\item framework creates classes from contributing plugin
		\item actual implementation inaccessible (dependency)
		\item created implicitly (default constructor)
		\item cannot access context -- use \ttt{ContextInjectionFactory}
			\singleindent{\alert{must be called in processor}}
	\end{itemize}
\end{frame}%>>>
\begin{frame}{Fragments}%<<<
	\begin{itemize}
		\item adds to \str{application model}
		\item GUI elements, commands, handlers, \ldots \eg
		\item extension (\ttt{org.eclipse.e4.workbench.model})
		\item GUI decomposition -- plugins for perspectives, parts, \ldots
	\end{itemize}
\end{frame}%>>>
\begin{frame}{Addons}%<<<
	\begin{itemize}
		\item model fragments $\Rightarrow$ use DI
		\item behavioral annotations
		\item typically \eg
			\begin{enumerate}
				\item provide context objects (services)
				\item listen to messages
				\item process extensions
			\end{enumerate}
	\end{itemize}
\end{frame}%>>>
\begin{frame}{Extensions vs Addons}%<<<
	\begin{enumerate}
		\item prefer \str{addons} (simpler, DI)
		\item \str{extensions} for existing extension points
		\item custom \str{extension points}
			\begin{itemize}
				\item model for future extensions
				\item multiple similar extensions
				\item (usually) used in one place
			\end{itemize}
	\end{enumerate}
	\begin{itemize}
		\item \str{extensions} must be explicitly processed
		\item plugin defines \str{extension point} and processes \str{extensions} via \str{addon} (puts results into \str{context})
	\end{itemize}
\end{frame}%>>>
\begin{frame}{Example Extension: Preference Pages}%<<<
	\begin{centerblock}Each plugin can provide its preference page.\end{centerblock}
	\begin{itemize}
		\item[]\str{Extension point}
			\begin{itemize}
				\item page id
				\item parent page id (optional)
				\item implementation extends \ttt{IPreferencePage}
			\end{itemize}
		\item[]\str{Command} w/ parameter: opened page id
		\item[]\str{Handler}
			\begin{enumerate}
				\item reads preference pages
				\item creates tree structure
				\item opens \ttt{PreferenceDialog} on given page
			\end{enumerate}
	\end{itemize}
\end{frame}%>>>

\breakframe

\begin{frame}{Miscellaneous}%<<<
	\begin{itemize}
		\item Eclipse 4 RCP (e3 was different)
		\item RCP uses SWT (heavyweight components)
		\item RAP $=$ RCP on web (in development)
	\end{itemize}
\end{frame}%>>>
\begin{frame}%<<<
	\frametitle{Summary}
	\begin{enumerate}
		\item \str{dependency injection}: I~want something, framework finds it
		\item \str{inversion of control}: I~write snippets, framework uses them
		\item \str{application model} -- IoC implementation, hierarchical
			\begin{itemize}
				\item windows, perspectives, parts, \ldots
				\item commands (actions), handlers (implementation)
			\end{itemize}
		\item \str{context} -- DI implementation, object storage
			\begin{itemize}
				\item hierarchical: application, windows, perspectives, parts
				\item persistent, changes not notified
			\end{itemize}
		\item \str{event service} -- global event sending
			\begin{itemize}
				\item for persistence, add context (uninitialized parts)
			\end{itemize}
		\item \str{extensions} -- metadata, executable extensions
		\item \str{fragments} -- add to application model
		\item \str{addons} -- provide services, listen to events, process extensions
	\end{enumerate}
\end{frame}%>>>
\begin{frame}{Sources}%<<<
	\begin{itemize}
		\item official page \url{https://wiki.eclipse.org/index.php/Rich_Client_Platform}

		\item tutorials from Lars Vogel
			\begin{itemize}
				\item Eclipse 4 RCP: The complete guide to Eclipse application development
				\item\url{http://www.vogella.com/tutorials/EclipseRCP/article.html}
			\end{itemize}

		\item slides and examples \url{https://github.com/martinbayer/com.cgi.example.e4.rcp}
	\end{itemize}
\end{frame}%>>>
\begin{simpleframe}
	\str{\Huge THANK YOU}

	\Large Any questions?
\end{simpleframe}

\end{document}
